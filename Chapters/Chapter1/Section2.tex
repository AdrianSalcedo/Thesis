\section{Control Systems}

In this section, we present the assumptions for the proof of existence of an optimal pair.

We assume that $U$ is a non-empty closed subset in $\mathbb{R}^n$ (it could be generally a separable complete metric space). For any initial pair $(t,x)\in \mathbb{R}_{+}\times\mathbb{R}^n$, we rewrite the control system here:

\begin{equation}\label{eq2.1}
	\left\{ \begin{array}{l}
	\dot{X}(s)=f(s,u(s),X(s)),\, s\in [t,\infty) \\
	X(t)=x.\\
	\end{array}
	\right.
\end{equation}

Let us begin with the following assumption:

(C1) The map $f:\mathbb{R}_{+}\times U\times \mathbb{R}^n\rightarrow \mathbb{R}^n$ is measurable and there exists a constant $L>0$ such that

$$\left\{ \begin{array}{l}
	|f(t,u,x_1)-f(t,u,x_2)|\leq L|x_1-x_2|,\, (t,u)\in \mathbb{R}_{+}\times U,\, x_1,x_2\in \mathbb{R}^n,\\
	|f(t,u,0)|\leq L,\,\mbox{for every}\,(t,u)\in \mathbb{R}_{+}\times U .\\
\end{array}
\right.$$

Note that this conditions imply:

$$|f(t,u,x)|\leq L(1+|x|),\,(t,u,x)\in\mathbb{R}_{+}\times U\times \mathbb{R}^n.$$

This condition is also usually called the lipschitz condition of the function $f$. A key feature of the above is that the bond of $|f(t,u,x)|$, depending on $|x|$, is uniform in $u$.

\begin{prop}\label{prop2.1.1}
	Let (C1) hold. Then, for any $(t,x)\in\mathbb{R}_{+}\times\mathbb{R}^n$, and $u(\cdot)\in \mathcal{U}[t,\infty)$, there exists a unique solution $X(\cdot)\equiv X(\cdot;t,x,u(\cdot))$ to the state equation (\ref{eq2.1}). Moreover, the following estimates hold:
	
	\begin{equation}\label{eq2.2}
	\left\{ \begin{array}{l}
	|X(s;t,x,u(\cdot))|\leq e^{L(s-t)}(1+|x|)-1, \\
	|X(s;t,x,u(\cdot))-x|\leq [e^{L(s-t)}-1](1+|x|),\, s\in (t,\infty],\, u(\cdot)\in \mathcal{U}[t,\infty).\\
	\end{array}
	\right.
	\end{equation} 
	
	Further, for any $t\in \mathbb{R}_{+}$, $x_1,x_2\in \mathbb{R}^n$, and $u(\cdot)\in \mathcal{U}[t,\infty)$,
	\begin{equation}\label{eq2.3}
	|X(s;t,x_1,u(\cdot))-X(s;t,x_2,u(\cdot))|\leq e^{L(s-t)}|x_1-x_2|,\,\mbox{for every}\, s\in [t,\infty).
	\end{equation}
\end{prop}

\begin{proof}
It suffices to prove our conclusion on any interval $[t,T]$ with $0\leq t<T<\infty$, Since $[t,\infty)=\bigcup_{T\geq t}[t,T]$.

For any $X(\cdot)\in C([t,T];\mathbb{R}^n)$, we define

$$[\mathcal{S}X(\cdot)](s):=x+\int_{t}^{s}f(r,u(r),X(r))dr,\,s\in [t,T].$$

Then for any $X_1(\cdot),X_2(\cdot)\in C([t,T];\mathbb{R}^n)$ and by condition (C1), we have

\begin{eqnarray*}
||[\mathcal{S}X_1(\cdot)]-[\mathcal{S}X_2(\cdot)]||_{C([t,T];\mathbb{R}^n)}&=&|| \int_{t}^{s}f(r,u(r),X_1(r))dr-\int_{t}^{s}f(r,u(r),X_2(r))dr||\\
&\leq& L||\int_{t}^{s} (X_1(r)-X_2(r)) dr||\\
&\leq& L||X_1(\cdot)-X_2(\cdot)||_{C([t,T];\mathbb{R}^n)}|t-s|.
\end{eqnarray*}

Define $\delta=|t-s|$, then

$$||[\mathcal{S}X_1(\cdot)]-[\mathcal{S}X_2(\cdot)]||_{C([t,T];\mathbb{R}^n)}\leq \delta L||X_1(\cdot)-X_2(\cdot)||_{C([t,T];\mathbb{R}^n)}.$$

Therefore, by choosing $\delta<\frac{1}{L}$, we see that $\mathcal{S}: C([t,T];\mathbb{R}^n)\rightarrow C([t,T];\mathbb{R}^n)$ is contractive. Hence, by Contraction Mapping Theorem (\ref{BFT}), the control system (\ref{eq2.1}) admits a unique solution on $[t,t+\delta]$. Repeating the argument, we can obtain that (\ref{eq2.1}) admits a unique solution on $[t,T]$.

Next, for the unique solution $X(\cdot)$ of (\ref{eq2.1}), we have

$$|X(s)|\leq |x|+L\int_{t}^{s}(1+|X(r)|)dr,\, s\in [t,T].$$

If we denote the right-hand side of the above by $\theta(s)$, then

$$\dot{\theta}(s)=L+L|X(s)|\leq L+L\theta(s),$$

which leads to 

$$\dot{\theta}(s)\leq e^{L(s-t)}|x|+L\int_{t}^{s}e^{L(s-r)dr}=e^{L(s-t)}|x|+e^{L(s-t)}-1,$$

thus, $|X(s)|\leq e^{L(s-t)}(1+|x|)-1.$ This gives the first estimate in (\ref{eq2.2}). Next, we will apply the first estimate
\begin{eqnarray*}
|X(s)-x| &=& |\int_{t}^{s}f(r,u(r),X(r))dr|\leq \int_{t}^{S}|f(r,u(r),X(r))|dr\\
&\leq& L\int_{t}^{s}(1+|X(r)|)dr\leq L\int_{t}^{s}e^{L(r-t)}(1+|x|)dr\\
&=& (1+|x|)[e^{L(s-t)}-1].
\end{eqnarray*}

This proves the second estimate in (\ref{eq2.2}). Finally, for $x_1,x_2\in \mathbb{R}^n$, let us denote $X_i(\cdot)=X(\cdot;t,x_i,u(\cdot))$. Then

$$|X_1(s)-X_2(s)|\leq L\int_{t}^{s} |X_1(r)-X_2(r)|dr.$$

By Gronwall's Inequality (\ref{GIP}), we obtain (\ref{eq2.3}).
\end{proof}

We can see that estimates (\ref{eq2.2}) are uniform in $u(\cdot)\in \mathcal{U}[t,T]$. This will play an interesting role later.